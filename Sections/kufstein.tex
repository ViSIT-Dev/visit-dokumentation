\section{Installationsprozess}\label{sec:installationsprozess}

\subsection{Infrastruktur}

Die ViSIT-Applikationen basieren auf der Server-Client-Architektur. Damit diese Applikationen installiert werden können, wird ein hausinternes Netzwerk (Intranet) und ein damit verbundener Server - lokaler Applikations-Server (kurz LAS) - benötigt. Die ViSIT-Applikationen sind in diesem Zusammenhang die Clients, welche über das Netzwerk mit dem lokalen Applikations-Server verbunden sind. Auf dem LAS ist das ViSIT-System installiert, welches über das Internet Zugang zum globalen ViSIT-Netzwerk hat.
Das ViSIT-System ist eine Ansammlung von mehreren kleinen Applikationen, welche parallel auf dem LAS laufen können. Jeder Client, auf welchem eine der ViSIT-Applikationen läuft, hat eigene Server-Software, welche auf dem LAS installiert ist und für die serverseitigen Berechnungen zuständig ist.

Die Applikationen wurden mit der IT-Technologie “Docker” erstellt. Mit Docker hat man die Möglichkeit, Anwendungen in sogenannten Containern auszuführen und diese Container können aufeinander aufbauen und miteinander kommunizieren. Im Gegensatz zu einer virtuellen Maschine, ist eine Docker-basierte Anwendung nur ein Prozess, der auf dem System ausgeführt wird. Es ist somit kein Gastbetriebssystem erforderlich, wie dies bei Virtuellen Maschinen der Falls ist. Container sind einfach konfigurierbare, abgeschlossene Einheiten, in welchen die Anwendung ausgeführt werden. 
Mit Docker können Linux-Container erstellt und verwendet werden können. Die erstellten Container sind eine Virtualisierung auf der Ebene des Betriebssystems. Durch das Erstellen von Containern, werden isolierte Linux-Systeme auf dem gleichen Host erzeugt. Diese Container können flexibel erstellt, bereitgestellt, kopiert und zwischen Umgebungen verschoben werden. Zweck dieser Container ist die Unabhängigkeit und die Fähigkeit, mehrere Prozesse und Applikationen getrennt voneinander betreiben zu können. Die Vorteile von Docker-Containern sind unter anderem Modularität und Versionsverwaltung. Modularität ermöglicht es, bei zum Beispiel einer Reparatur oder Aktualisierung einer Applikation, nur einen Teil dieser Applikation außer Betrieb zu nehmen, ohne die gesamte Applikation außer Betrieb nehmen zu müssen. Docker bietet eine eingebaute Versionsverwaltung, welche es erlaubt, den aktuellen Stand eines Containers in ein sogenanntes Image zu sichern. Somit ist es möglich, die unterschiedlichen Zustände eines Images in einer Historie nachzuverfolgen. Ein Image ist ein Speicherabbild eines Containers und es besteht aus mehreren Layern, welche schreibgeschützt sind und somit nicht verändert werden können. Ein Layer ist wiederum ein Teil eines Images und enthält einen Befehl oder eine Datei, welche dem Image hinzugefügt wurde. Aufgrund dieser Layer kann die ganze Historie eines Images nachvollzogen werden.

\subsection{Projekt auf dem LAS installieren}

Als erster Schritt muss die Datenbank für die Applikation angelegt werden. Wie oben erklärt, wurde für das ViSIT-Projekt Docker verwendet. Damit gespeicherte Daten auch außerhalb eines Containers abgelegt oder in einem anderen Container eingebunden werden können, werden sogenannte Volumes erstellt. Volumes haben viele Vorteile, vor allem aber sind sie einfacher zu sichern oder zu migrieren. Volumes funktionieren sowohl auf Linux- als auch auf Windows-Containern.
Im ersten Schritt wird ein Volume mit der Datenbank auf dem lokalen Rechner im Terminal mit dem Kommando
\begin{lstlisting}[style=MyBashStyle]
docker volume create visit-database
\end{lstlisting} erstellt. Einen eigenen Volume benötigt man deshalb, weil die dort abgelegten Daten permanent gespeichert werden müssen - würde z.B.: der Container gelöscht oder beendet werden - dann wären die nur im Docker Container gespeicherten Daten ebenfalls gelöscht werden. Damit dies nicht passieren kann, werden die Daten parallel lokal auf dem Rechner gespeichert.
Damit Dateien zwischen Geräten in einem lokalen Netzwerk oder zwischen entfernten Geräten über das Internet synchronisiert werden können, wird eine Datensynchronisation mit Peer-to-Peer-Übertragung benötigt. Dies wird im ViSIT-Projekt mit Syncthing realisiert und auch dafür muss ein eigener Volume lokal auf dem Rechner erstellt werden. Dies geschieht mit \begin{lstlisting}[style=MyBashStyle]
docker volume create visit-syncthing
\end{lstlisting}-Befehl, welcher ebenfalls im Terminal ausgeführt wird.
Als nächster Schritt wird das gesamte ViSIT-Projekt von GitHub mittels

\begin{lstlisting}[style=MyBashStyle]
docker run -d --name visit -p 80:80 -p 22000:22000 -p 21027:21027
-v visit-syncthing:/var/syncthing
-v s:/p2p/visit:/var/p2p
-v visit-database:/var/lib/mysql
--restart unless-stopped visitapp/maincontainer \end{lstlisting}

geklont. Beim erstmaligen Starten benötigt der Vorgang länger, da das Projekt aus dem Git Repository sowie das Appbundle (https://github.com/ViSIT-Dev/appbundle) heruntergeladen werden.\\

Erklärung der einzelnen Befehle:\\
\begin{lstlisting}[style=MyBashStyle]
docker run -d --name visit -p 80:80 -p 22000:22000 -p 21027:21027 
\end{lstlisting}

\begin{lstlisting}[style=MyBashStyle]
docker run 
\end{lstlisting}
 startet den Container und mit den mit den Parametern \begin{lstlisting}[style=MyBashStyle]
-d 
\end{lstlisting} gibt man an, dass der Container im Hintergrund dauerhaft laufen soll (Daemonmode). Weiters wird mit \begin{lstlisting}[style=MyBashStyle]
--name visit
\end{lstlisting} der Name des Containers festgelegt, in diesem Fall heißt der Container visit. Der Container kann im weiteren Verlauf auch über diesen Namen angesprochen werden.
Mit dem Parameter
\begin{lstlisting}[style=MyBashStyle]
-p 80:80
\end{lstlisting} werden die Ports vom Host an den Container gebunden. Hier wird der lokale Hostport 80 auf den Containerport 80 gemappt. Die weiteren Ports 
\begin{lstlisting}[style=MyBashStyle]
-p 22000:22000 -p 21027:21027 
\end{lstlisting} werden für das Syncthing und für das Peer to Peer-Netzwerk benötigt.
Als nächstes folgt der Befehl 
\begin{lstlisting}[style=MyBashStyle]
-v visit-syncthing:/var/syncthing
\end{lstlisting}
Mit dem Parameter \begin{lstlisting}[style=MyBashStyle]
-v
\end{lstlisting}
 wird ein Verzeichnis (Volume) auf dem Hostrechner zu einem Verzeichnis innerhalb des Containers verbunden, auf diese Weise werden die Daten persistent gespeichert, das heißt, dass ein Ordner auf dem Hostsystem auf einen Ordner im Container gemappt wird. Das bedeutet, dass die Daten in beiden Ordnern immer inhaltsgleich sind. Ohne dem Mapping zu einen Ordner auf dem Hostsystem, wären alle Daten aus dem Docker Container, wenn dieser Container gelöscht wird, ebenfalls gelöscht. Um die Daten persistent, also dauerhaft zu speichern, wird immer ein Ordner im Hostsystem mit dem entsprechenden Ordner im Docker Container gemappt.
Zuerst wird das Verzeichnis auf dem Hostrechner angegeben, hier \begin{lstlisting}[style=MyBashStyle]
visit-syncthing
\end{lstlisting}
und nach dem Doppelpunkt steht das Verzeichnis innerhalb des Containers, hier \begin{lstlisting}[style=MyBashStyle]
/var/syncthing
\end{lstlisting}

Im nächsten Teil des Befehls \begin{lstlisting}[style=MyBashStyle]
-v s:/p2p/visit:/var/p2p
\end{lstlisting}
 wird ebenfalls zuerst das Verzeichnis auf dem Hostrechner angegeben, \begin{lstlisting}[style=MyBashStyle]
s:/p2p/visit
\end{lstlisting} und dann das Verzeichnis innerhalb des Containers \begin{lstlisting}[style=MyBashStyle]
/var/p2p
\end{lstlisting}
Im nächsten Befehl \begin{lstlisting}[style=MyBashStyle]
-v visit-database:/var/lib/mysql
\end{lstlisting} geht es um die Verbindung zur Datenbank. Hier wird ebenfalls zuerst das Verzeichnis auf dem Hostrechner angegeben \begin{lstlisting}[style=MyBashStyle]
visit-database
\end{lstlisting} und nach dem Doppelpunkt steht das Verzeichnis innerhalb des Containers \begin{lstlisting}[style=MyBashStyle]
/var/lib/mysql
\end{lstlisting}
Zuletzt wird mittels \begin{lstlisting}[style=MyBashStyle]
--restart unless-stopped visitapp/maincontainer
\end{lstlisting} dem System mitgeteilt, dass der Docker Container \begin{lstlisting}[style=MyBashStyle]
visitapp/maincontainer
\end{lstlisting} automatisch gestartet werden soll außer, wenn er manuell oder anderweitig gestoppt wird.

Wenn der Vorgang abgeschlossen ist, kann über Lokalhost im Browser unter \textbf{localhost:80/typo3/} das Backend aufgegerufen werden (siehe Abbildung \ref{img:typo_3_login}). Das erstmalige einloggen in das Backend (TYPO3) erfolgt mit dem \textbf{Benutzername: admin} und \textbf{Passwort: visit-admin}.

\begin{figure}[ht!]
\centering
\includegraphics[width=12cm]{Figures/paula/typo_3_login.png}
\caption{Das Login-Fenster für TYPO3 im Browser}
\label{img:typo_3_login}
\end{figure}



\section{TYPO3}
\subsection{Allgemein}

TYPO3 ist ein freies Content-Management-System für Webseiten, es wird in Frontend und Backend getrennt. Als Frontend wird die Präsentationsebene bezeichnet, das ist der Teil einer Applikation, den der Betrachter sehen kann. Als Backend hingegen, bezeichnet man die Datenzugriffsebene, das ist der Teil einer Applikation, welcher nicht für den Besucher sichtbar ist. Das Backend ist der Verwaltungsbereich einer Webseite. TYPO3 wird auf einem Webserver installiert und über den Webbrowser benutzt.

Das Backend ist die Datenzugriffsebene, dieser Teil ist für den Endbenutzer nicht sichtbar. Es beinhaltet die Programmierung einer Applikation und den Administrationsbereich. Im Gegensatz dazu das Frontend, das ist die tatsächliche Webseite, die der Endbenutzer im Browser sieht, also die Benutzeroberfläche.

\subsection{Login}

Damit niemand unbefugter im Frontend sowie Backend etwas verändern kann, muss man sich zuerst ins Backend einloggen. Dies geschieht über den Aufruf der Domain \textbf{localhost:80/typo3/} im Webbrowser (siehe Abbildung \ref{img:typo_3_login}).

\begin{figure}[ht!]
\centering
\includegraphics[width=8cm]{Figures/paula/login_TYPO3.png}
\caption{Das Login-Fenster für TYPO3}
\label{img:typo_3_logIn}
\end{figure}

Im Login-Fenster kann der Benutzername sowie das Passwort eingetragen werden (siehe Abbildung \ref{img:typo_3_logIn}). Beim ersten Login ist der \textbf{Benutzername: admin} und das \textbf{Passwort: visit-admin}, dieser muss in weiterer Folge verändert werden. Mehr dazu siehe Anpassung.
Nach einem erfolgreichen Login wird das Backend mit den dazugehörigen Modulen im Browser geladen.

\begin{figure}[ht!]
\centering
\includegraphics[width=12cm]{Figures/paula/aufbau_TYPO3.png}
\caption{Aufbau des TYPO3-Backends}
\label{img:typo_3_backend}
\end{figure}

\subsection{Aufbau von TYPO3}

Das TYPO3-Backend besteht aus einem Kopfbereich (grün eingerahmt) und einem Hauptbereich (rot eingerahmt), welcher aus drei Spalten besteht (siehe Abbildung \ref{img:typo_3_backend}). Im Kopfbereich kann der Administrator seine TYPO3-Benutzerenstellungen konfigurieren. Im Hauptbereich werden Webdokumente bearbeitet. Das TYPO3-Backend wird von links nach rechts abgearbeitet.

\subsubsection{Kopfleiste}

Die Kopfleiste bietet die Möglichkeit, die im TYPO3 Backend gespeicherten Lesezeichen aufzurufen (Stern-Symbol), den TYPO3 Cache der gesamten Webseite zu leeren (Blitz-Symbol) sowie Hilfe und Dokumentationen (Fragezeichen) zu TYPO3 aufzurufen. Das vierte Symbol zeigt die wichtigsten Systeminformationen. Mit einem Klick auf den Benutzernamen, in der Grafik “admin”, öffnet sich ein Kontext-Menü mit der Möglichkeit Einstellungen an seinem Benutzer vorzunehmen oder sich aus dem TYPO3 Backend auszuloggen. Rechts neben dem Benutzer befindet sich das Suchfeld, mit dem sich das gesamte TYPO3 Backend durchsuchen lässt.

\subsubsection{Die Spalten des Hauptbereichs}

\textbf{Linke Spalte:} \textit{Modulleiste} (blau eingerahmt), hier kann das Modul ausgewählt werden, welches bearbeitet werden soll (siehe Abbildung \ref{img:typo_3_backend}).\\

\textbf{Mittlere Spalte:} \textit{Seitenbaum} (gelb eingerahmt), hier wird die zu bearbeitende TYPO3-Seite ausgewählt. Der Seitenbaum ist das zentrale Element, wenn es darum geht sich durch die Webseite zu navigieren. Hier wird der Aufbau und die Seitenhierarchie der Webseite in einer Struktur abgebildet, die der Ordnerstruktur ähnlich ist. Einzelne Seiten können Unterseiten enthalten, die im Seitenbaum eingerückt dargestellt werden (siehe Abbildung \ref{img:typo_3_backend}).\\

\textbf{Rechte Spalte:} \textit{Arbeitsbereich} (violett eingerahmt), hier wird am ausgewählten TYPO3-Objekt gearbeitet (siehe Abbildung \ref{img:typo_3_backend}).

\subsection{Konfiguration des Backends mit TYPO3}

Die für die Applikationen benötigten TYPO3 Extensions werden automatisch installiert, sollte eine weitere Extension benötigt werden, befindet sich eine Anleitung für die Installation in diesem Abschnitt. Extensions sind optionale Software-Komponenten, also Zusatzmodule, die eine bestehende Software erweitern.\\
Installation von TYPO3 Extensions: Dazu wird in der linken Spalte zuerst das Modul “Extensions” ausgewählt. Dann erscheinen im Hauptfenster verschiedene Extensions, welche alphabetisch gelistet sind. Bei der Erstinstallation werden folgende Extensions (unten ist der Key angegeben, welcher sich in der mittleren Spalte befindet) automatisch installiert (siehe Abbildung \ref{img:extensions}):

\begin{itemize}
    \item visit\_tablets
    \item scheduler
    \item tstemplate
    \item fluid\_styled\_content
    \item setup
\end{itemize}

\begin{figure}[ht!]
\centering
\includegraphics[width=12cm]{Figures/paula/extensions.png}
\caption{Installation der Extensions}
\label{img:extensions}
\end{figure}

Mittels einem Klick auf das Würfelsymbol mit einem Plus werden die oben angegebenen Extensions der Reihe nach aktiviert (siehe Abbildung \ref{img:extensions}). Die aktivierten Extensions erscheinen dann als auswählbare Module in der linken Spalte.
Optional kann im nächsten Schritt die Sprache Deutsch installiert werden, sonst ist die Hauptsprache Englisch. Um die Sprache zu installieren, wird in der linken Spalte unter den ADMIN TOOLS “Languages” ausgewählt (siehe Abbildung \ref{img:sprache_aendern}).

\begin{figure}[ht!]
\centering
\includegraphics[width=12cm]{Figures/paula/sprache_aendern.png}
\caption{Änderung der Sprache}
\label{img:sprache_aendern}
\end{figure}

Im Hauptfenster erscheinen nach dem Klick die unterstützten Sprachen, hier “German” suchen und zuerst mittels einem Klick auf das Plus-Symbol links von der Sprache die Sprache aktivieren, dabei erscheint oben rechts eine grüne Meldung mit “Success, language was successfully activated.”. Als nächstes muss die aktivierte Sprache mittels Klick auf das Download-Symbol rechts von der Sprache heruntergeladen werden. War der Download erfolgreich, so erscheint oben rechts eine grüne Meldung mit “Success. The translation update has been successfully completed.”.

\subsection{Anpassung}

Im Kopfbereich können die TYPO3-Benutzereinstellungen konfiguriert werden. Dazu wird im Kopfbereich oben rechts zuerst der Benutzer ausgewählt. Bei der Erstinstallation ist es der “admin”, dabei wird ein Menü aufgeklappt, aus welchem die “User settings” ausgewählt werden (siehe Abbildung \ref{img:benutzereinstellungen}).

\begin{figure}[ht!]
\centering
\includegraphics[width=12cm]{Figures/paula/benutzereinstellungen.png}
\caption{Konfiguration der Benutzereinstellungen}
\label{img:benutzereinstellungen}
\end{figure}

Jetzt erscheinen im Hauptbereich die User Settings, welche in dieser Maske konfiguriert werden können. Jetzt kann zuerst die Sprache umgestellt werden. Dies kann gleich im ersten Raster “Personal data”, im unteren Bereich unter Languages geändert werden (siehe Abbildung \ref{img:benutzereinstellungen_sprache}). Hier kann die heruntergeladene Sprache mittels Dropdown ausgewählt werden. Damit die Auswahl auch gespeichert und angewendet wird, muss auf das Speicher-Symbol (Diskette) ganz oben links im Hauptfenster geklickt werden. Nur durch diesen Klick werden die User Settings upgedated und die Sprache auch angewendet. Jetzt erscheinen oben im Hauptfenster drei Meldungen. Die grüne Meldung besagt, dass die Settings upgedated wurden. Die blaue Meldung sagt, dass die Seite (localhost:80/typo3/) neu geladen werden muss, um die Veränderungen zu aktivieren. Die rote Meldung sagt, dass das neue Passwort nicht upgedated wurde, da es nicht zweimal eingegeben wurde.

\begin{figure}[ht!]
\centering
\includegraphics[width=12cm]{Figures/paula/benutzereinstellungen_sprache.png}
\caption{Änderung der Benutzereinstellungen und der Sprache}
\label{img:benutzereinstellungen_sprache}
\end{figure}

Als nächstes wird das Passwort verändert. Dazu wird die Registerkarte “Password” ausgewählt (siehe Abbildung \ref{img:aenderung_passwort}). Jetzt erscheint das zuvor eingegebene Passwort “visit-admin” als eine Punkte-Kette in der ersten Zeile, hier kann das Passwort mit einem neuen Passwort überschrieben werden. Gleiches Passwort wird in der darunter liegenden Zeile nochmals eingegeben. Damit die Änderungen gespeichert werden, wird wieder oben links das Speichern-Symbol geklickt. Ab jetzt werden auch die Änderungen der Sprache angewendet und alles wird auf Deutsch angezeigt. Mit diesem Schritt ist das Backend fertig vorbereitet.

\begin{figure}[ht!]
\centering
\includegraphics[width=12cm]{Figures/paula/aenderung_passwort.png}
\caption{Änderung des Passworts}
\label{img:aenderung_passwort}
\end{figure}

\subsection{Hinzufügen einer Applikation aus dem App-Bundle}

Dazu wird in der linken Spalte “Seite” ausgewählt. Jetzt kann dem ViSIT App Container eine Seite hinzugefügt werden. Zuerst muss auf das oben ganz links befindlichen Seiten-Symbol geklickt werden, dann erscheint eine Auswahl an möglichen Aktionen. Hier das erste leere Seite-Symbol anklicken und auf den darunter befindlichen ViSIT App Container ziehen und darüber loslassen, anschließend kann der Seite ein Name gegeben werden (siehe Abbildung \ref{img:neue_seite_hinzufuegen}).

\begin{figure}[ht!]
\centering
\includegraphics[width=12cm]{Figures/paula/neue_seite_hinzufuegen.png}
\caption{Hinzufügen einer neuen Seite}
\label{img:neue_seite_hinzufuegen}
\end{figure}

Mittels Rechtsklick auf die soeben erstellte Seite erscheint unter der Seite ein weiteres Menü, aus diesem dann “Bearbeiten” auswählen. Danach kann rechts die Seite konfiguriert werden.\\
Im nächsten Schritt muss das Verhalten der Seite konfiguriert werden. Dazu den Raster “Verhalten” anklicken und unter “Sonstige” “Als Anfang der Website benutzen” aktivieren.
Dann den Raster “Zugriff” auswählen und unter “Sichtbarkeit” “Seite” deaktivieren.
Nachdem die Änderungen durchgeführt wurden, müssen diese gespeichert werden. Dazu muss auf das Speicher-Symbol oben auf der Hauptseite geklickt werden. Danach erscheint ein Weltkugel-Symbol neben der soeben erzeugten Seite im linken Teil des Hauptfensters.

\subsection{Erzeugung des Layouts}

Um das Layout der Seite zu definieren, muss auf die soeben erzeugte Seite geklickt werden (siehe Abbildung \ref{img:layout_erzeugung}).

\begin{figure}[ht!]
\centering
\includegraphics[width=12cm]{Figures/paula/layout_erzeugung.png}
\caption{Erzeugung des Layouts der neu erstellten Seite}
\label{img:layout_erzeugung}
\end{figure}

Im rechten Teil des Hauptfensters erscheinen vier Möglichkeiten der Inhaltspositionierung. Für die ViSIT-Applikationen wird die normale Inhaltspositionierung benötigt.  Um weitere Konfiguration durchzuführen, unter “Normal” auf das das Inhalts-Symbol klicken und im Raster “Plug-Ins” auswählen, hier können die Plugins für die jeweilige ViSIT-Applikation ausgewählt werden (siehe Abbildung \ref{img:auswahl_plugins}).

\begin{figure}[ht!]
\centering
\includegraphics[width=12cm]{Figures/paula/auswahl_plugins.png}
\caption{Auswahl der Plug-Ins}
\label{img:auswahl_plugins}
\end{figure}

\subsection{Das Karten-Plug-In}

Im Raster “Plug-Ins” die “Karte - Dieses Plugin einfügen um eine Karte anzuzeigen” auswählen und oben auf das Speicher-Symbol klicken, damit die Änderungen gespeichert werden. Nach dem Speichern kann die Seite mit dem X-Symbol über der Überschrift geschlossen werden. Danach erscheint die Übersicht über die erzeugte Seite, hier sieht man, dass das Karten-Plugin eingebunden wurde (siehe Abbildung \ref{img:einbindung_plugins}).

\begin{figure}[ht!]
\centering
\includegraphics[width=12cm]{Figures/paula/einbindung_plugin.png}
\caption{Einbindung eines Plug-Ins}
\label{img:einbindung_plugins}
\end{figure}

Wenn jetzt die soeben erstellte Seite in der linken Spalte des Hauptfensters, also da wo die Weltkugel ist, mit Rechtsklick ausgewählt, kommt ein Dropdown-Menü. Jetzt den ersten Eintrag “Ansehen” aus der Liste auswählen und die Seite kann im Browser angesehen werden.

\subsection{Erstellung eines Templates}

Wenn ein neuer Raum hinzugefügt wird, wird ein neuer Webroot benötigt, dieser wird mittels Template erzeugt und stellt den Seitenanfang der Webseite dar. Sollen mehrere gleiche Applikationen laufen, dann wird für jede einzelne Applikation ein eigenes Template benötigt.
Für die Darstellung der Inhalte auf der Webseite werden Templates verwendet. Ein Template ist eine Design- und Formatierungsvorlage für ein Dokument, es ist das Grundgerüst, welches mit Inhalten gefüllt werden muss//.
Um ein Template in TYPO3 zu erstellen, muss im ersten Schritt unter WEB das “Template” aus der Modul-Liste auf der linken Seite ausgewählt werden. Danach erscheinen die Template-Werkzeuge in der rechten Hälfte des Hauptfensters, hier kann “Template für neue Website erstellen” ausgewählt werden. Jetzt kann in der Werkzeugleiste des Hauptbereichs das Dropdown-Feld aufgemacht und “Info/Bearbeiten” ausgewählt werden. In der Übersicht im Hauptbereich erscheinen die wichtigsten Template-Informationen. Danach “Vollständigen Template-Datensatz bearbeiten” auswählen. Hier kann im Raster “Allgemeines” der Titel des Templates hinzugefügt werden, des weiteren muss der Inhalt aus “Setup” gelöscht werden.\\
Danach ins Raster “Enthält” wechseln, hier können verschiedene Objekte aus der rechten Spalte “Verfügbare Objekte” in die linke Spalte “Ausgewählte Objekte” verschoben werden, hier muss jedoch auf die Reihenfolge dieser Objekte geachtet werden. Hier zuerst auf “Fluid Content Elements (fluid\_styled\_content)” klicken, dann wandert dieses Objekt in die linke Spalte. Das gleiche mit dem Objekt “tablets (visit\_tablets)”. Jetzt befinden sich beide Objekte in der linken Spalte unter “Ausgewählte Objekte”. Damit diese Änderungen gespeichert werden, muss wieder auf das Speichern-Symbol über der Überschrift im Hauptbereicht geklickt werden. Wenn die Webseite auf dem localhost:80/ aufgerufen wird, erscheint die Karte.


\section{Karten-Applikation}

\subsection{Einpflegen der Daten in die Karten-Applikation}

Dazu aus der Modulleiste links unter der Obergruppe VISIT TABLET die Karte auswählen. Im linken Teil des Hauptfensters ist der Seitenbaum zu sehen und rechts befindet sich die Kartenübersicht. Oben links im rechten Teil des Hauptfensters befindet sich ein Menü-Button, wird dieser angeklickt, wird eine weitere dunkelblaue Spalte zwischen dem Seitenbaum und dem Arbeitsbereich im Hauptfenster sichtbar (siehe Abbildung \ref{img:kartenapp}).

\begin{figure}[ht!]
\centering
\includegraphics[width=12cm]{Figures/paula/kartenapp.png}
\caption{Leere Kartenübersicht}
\label{img:kartenapp}
\end{figure}

\subsection{Erstellung der Startseite für die Karten-Applikation}

Dazu in der dunkelblauen Leiste “Einstellungen” auswählen (siehe Abbildung \ref{img:startseite_karte}).

\begin{figure}[ht!]
\centering
\includegraphics[width=12cm]{Figures/paula/startseite_karte.png}
\caption{Erstellung der Startseite für die Karten-Applikation}
\label{img:startseite_karte}
\end{figure}

Die Startseite wird dem Besucher als erstes angezeigt, auf dieser kann der Besucher die gewünschte Sprache auswählen. Damit das Design des Textes immer gleich aussieht, gibt es unter \url{https://github.com/ViSIT-Dev/appbundle} in der README.md ein Beispiel für die Startseite der Tablets (siehe Abbildung \ref{img:github_link}).

\begin{figure}[ht!]
\centering
\includegraphics[width=12cm]{Figures/paula/github_link.png}
\caption{Text für die Startseite der Applikationen, zu finden auf \url{https://github.com/ViSIT-Dev/appbundle}}
\label{img:github_link}
\end{figure}

Für jede Sprache wird ein Titel sowie der Impressumstext benötigt. Jetzt werden die beiden Texte aus der zuvor genannten Github-Seite benötigt (siehe Abbildung \ref{img:github_link}). Der erste Text ist der Starttext, dieser beinhaltet die HTML-Elemente Überschrift, Paragraph und Buttons über welche die gewünschte Sprache gewählt werden kann. Die einzelnen Texte in den Tags können mit dem gewünschten Text überschrieben werden (siehe Abbildung \ref{img:befuellte_startseite_karte}).\\
Wenn weitere Sprachen außer Deutsch und Englisch verfügbar sind, können weitere Sprachauswahl-Buttons durch das Markieren des gesamten <button>-Tags ausgewählt werden, dann kopieren und darunter einfügen, erstellt werden. Zwei Sachen müssen beachtet werden: einerseits muss in der \textit{onclick="initMap('...')}-Methode die der Sprache entsprechende ID eingegeben werden und für den Button der Pfad für die Flagge im Image-Tag angegeben werden. Dazu muss zuvor im das benötigte Flaggen-Icon vorzugsweise im PNG-Format im entsprechenden Ordner gespeichert werden und der Pfad angepasst werden\\ \textit{src="/typo3/sysext/core/Resources/Public/Icons/Flags/PNG/DE.png">}. 

\begin{figure}[ht!]
\centering
\includegraphics[width=12cm]{Figures/paula/befuellte_startseite_karte.png}
\caption{Erstellung der Startseite für die Karten-Applikation}
\label{img:befuellte_startseite_karte}
\end{figure}

\section{Neues Kartenelement hinzufügen}

Mit einem Klick auf “Neu” kann ein neues Kartenelement - Point of Interest - hinzugefügt werden (siehe Abbildung \ref{img:neues_kartenelement_hinzufuegen}).

\begin{figure}[ht!]
\centering
\includegraphics[width=12cm]{Figures/paula/neues_kartenelement_hinzufuegen.png}
\caption{Ein neues Kartenelement hinzufügen}
\label{img:neues_kartenelement_hinzufuegen}
\end{figure}

Ein neues Kartenelement - Point of Interest - benötigt eine Überschrift, eine Unterüberschrift ist optional, einen Text auf der Flagge und eine Beschreibung. Optional können auch weitere Medien hinzugefügt werden. Die geografische Position kann entweder über den Längen- und Breitengrad manuell eingetippt werden oder mittels setzen der Stecknadel auf die Karte, dann werden die Längen- und Breitengrade dieser Stecknadel übernommen. Ist alles vollständig ausgefüllt, kann die Eingabe mit “Anlegen” am Seitenende gespeichert werden. Nach dem Klick gelangt man zu der Kartenübersicht, wo alle eingefügten Elemente angeführt sind, jedes dieser Elemente kann sowohl nochmals bearbeitet oder auch wieder gelöscht werden.\\
Klickt man im Seitenbaum mit der rechten Maustaste auf Karte, dann kann man die angelegten Kartenelemente im Browser anzeigen lassen.\\
Jedes Kartenelement muss sowohl auf Deutsch als auch auf Englisch angelegt werden.


\cite{anno4j1}